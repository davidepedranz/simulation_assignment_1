\section{Mean Confidence Interval}
\label{sec:confidence}

The data set contains a sinusoidal noise, which makes the samples sample not identically distributed.
In order to compute a small confidence interval with a high probability, we need to transform the data set to obtain \ac{i.i.d.} samples.
The used technique is the batch means.

We observe that the mean operator is linear. This allows to compute the mean of the original data set by dividing the samples in batches of some fixed size, computing the mean of each batch and finally the mean of the means.

If we take the size of the batch as a multiple of the wavelength of the sinusoid and compute the batches, we obtain a population of random variables \ac{i.i.d.}, since the sum of any sinusoidal signal over a single period is zero.
Since the size is still high (\num{300} elements), we can treat the new population as normally distributed.
We computed two confidence intervals for a confidence of \num{95}\% and \num{99}\%:

\begin{equation}
    \begin{split}
        P[2.94 < \mu < 3.06] &= 0.95 \\
        P[2.92 < \mu < 3.08] &= 0.99    
    \end{split}
    \label{eq:confidence}
\end{equation}
