\section{Third Parameter - Amplitude $A$}
\label{sec:amplitude}

The amplitude $A$ of the sinusoid can be computed in a similar way.
\cref{eq:autocorrelation} describe the relationship between the original sine signal and its autocorrelation function.
To estimate the amplitude of the autocorrelation function $A_{cos}$, we take the average of the absolute value of the theoretical highest and lowest points of each period of the cosine.
Formally:

\begin{equation}
    \begin{gathered}
        n = \Big\lfloor \frac{2 \cdot lag}{N} \Big\rfloor, \\
        A_{cos} = \frac{1}{n} \sum_{\tau=1}^{n} |R(\tau + lag)| + |R(\tau + 2 \cdot lag)| = 0.81
    \end{gathered}
\end{equation}

As for the case of the frequency, we take the mean over all peaks to reduce the noise due to the variances of $Y$ and $Z$.

Then, we can easily derive the estimation of the original amplitude $A$:

\begin{equation*}
    \hat{A} = \sqrt{2 \cdot A_{cos}} = 1.27.
\end{equation*}