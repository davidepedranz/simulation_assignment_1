\section{Conclusion}
\label{sec:conclusion}

\begin{table}
	\centering
	\caption{Summary of the estimated parameters}
	\label{tab:summary}
	\begin{tabular}{l>{\raggedright}p{3cm}c}
    	\toprule
    		\multicolumn{2}{l}{parameter} & \multicolumn{1}{c}{value} \\
    	\midrule
    		$\mu$    & mean of the logistic $Y$  & 3.00              \\
    		$f$      & frequency of the sinusoid & \SI{200}{\hertz}  \\
    		$A$      & amplitude of the sinusoid & 1.27              \\
    		$s$      & scale of the logistic     & 0.49              \\
    		$\sigma$ & s.d. of the Gaussian      & 0.96              \\
    	\bottomrule
	\end{tabular}
\end{table}


In this work, we solved the problem of estimating the parameters of a given model using a sample of measures.
This type of analysis should be used for every measure that requires some reliability, since every probe is subject to somer random noise.

The estimated values for each parameter are summarized in \cref{tab:summary}.
The confidence intervals defined in \cref{eq:confidence} are reported below for completeness:
\begin{align*}
    P[2.94 < \mu < 3.06] &= 0.95 \\
    P[2.92 < \mu < 3.08] &= 0.99
\end{align*}

First, we estimated the value for the mean parameters, which was not affected by the others.
Secondly, we used the auto correlation function to isolate and remove the sinusoidal noise.
Thirdly, we exploited the different shapes and moments of the Gaussian and the logistic distribution to separate them from each other.
Finally, we computed a confidence interval for the mean using the technique of the batch means.
