\section{Introduction}
\label{sec:introduction}
Circuitry are widely used in every electronic device.
In order for those devices to work properly, high quality circuitry is needed.
The quality of a circuitry can be evaluated measuring some of its properties.
Unfortunately, each measurement is subject to some random noise or bias.
In order to achieve reliable results, the stochastic nature of the measuring process must be taken into account.

In this case, we have a circuitry which is known to generate \ac{i.i.d.} samples taken from a population with a logistic distribution.
The electronic probe used for the measures is known to introduce a white Gaussian noise and a sinusoidal bias.
The actual measures can thus be described as:

\begin{equation}
    \begin{gathered}
        x_{t} = A sin(2 \pi f t) + Y + Z, \\
        Y \sim Logis(\mu, s), \; Z \sim N(0, \sigma_{z}).
    \end{gathered}
    \label{eq:original-sample}
\end{equation}

Our aim is to estimate the five parameters of the model given the measures data set.
In particular, we are interested in the mean $\mu$ of the logistic distribution $Y$.
In addition, a confidence interval for the $\hat{\mu}$ is required.

The given data set contains $N = \num{60000}$ points from a sampling done at $ F = \SI{20}{\kilo\hertz}$ for \SI{3}{\second}.
We assume the frequency $f$ of the sinusoid to be an integer divisor of the sampling frequency and to be low, in the order of ten/hundreds of \SI{}{\hertz}.

We will estimate one parameter at a time, use it to clean the samples, then focus on the next one until the last one.
In particular, we will start from $\mu$, then estimate $f$ and $A$, and finally $s$ and $\sigma$.
As the last step, we will estimate the confidence interval for $\hat{\mu}$.
